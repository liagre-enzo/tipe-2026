%%%%%%%%%%%%%%%%%%%%%%%%
%
%   TIPE Liagre Enzo
%   MPI Faidherbe
%
%%%%%%%%%%%%%%%%%%%%%%%%



\documentclass{beamer}
\usetheme{Berkeley}
\usepackage[french]{babel}
\usepackage[T1]{fontenc}
\usepackage[autolanguage]{numprint}

\title{TIPE}
\author{L'émulation et la conservation des logiciels}
\institute{}
\date{\scriptsize LIAGRE Enzo}

\begin{document}
    
    \begin{frame}
        \titlepage\
    \end{frame}

    
    %\begin{frame}{Tables des matières}
    %    \tableofcontents
    %\end{frame}
    
    \section{Présentation du sujet}
    \subsection{Introduction}
    \begin{frame}{Introduction}
        \begin{definition}[Émulation]
            L'émulation est le processus par lequel une application reproduit le fonctionnement d'une machine ou d'un autre logiciel.
        \end{definition}
        \color{white} Exemples:
        \begin{itemize}
            \item[\color{white}] \color{white} Un émulateur de terminal
            \item[\color{white}] \color{white} Une machine virtuelle
        \end{itemize}
    \end{frame}

    \begin{frame}{Introduction}
        \begin{definition}[Émulation]
            L'émulation est le processus par lequel une application reproduit le fonctionnement d'une machine ou d'un autre logiciel.
        \end{definition}
        Exemples:
        \begin{itemize}
            \item Un émulateur de terminal
            \item[\color{white}] \color{white} Une machine virtuelle
        \end{itemize}
    \end{frame}

    \begin{frame}{Introduction}
        \begin{block}{Émulation}
            L'émulation est le processus par lequel une application reproduit le fonctionnement d'une machine ou d'un autre logiciel.
        \end{block}
        Exemples:
        \begin{itemize}
            \item Un émulateur de terminal
            \item Une machine virtuelle
        \end{itemize}
    \end{frame}

    \subsection{Principe}
    \begin{frame}{Principe}
        On distingue deux méthodes pour l'émulation.
        \begin{enumerate}
            \item[\color{white}] \color{white} L'émulation de bas niveau (Low-level emulation)
            \begin{itemize}
                \item[\color{white}] \color{white} $\hookrightarrow$ On reproduit le fonctionnement de la machine en entier.
            \end{itemize}
            \item[\color{white}] \color{white} L'émulation de haut niveau (High-level emulation)
            \begin{itemize}
                \item[\color{white}] \color{white} $\hookrightarrow$ On reproduit ce que la machine permet.
            \end{itemize}
        \end{enumerate}
        {\color{white} passage de ligne}

        \color{white} On s'intéresse dans un premier temps à l'émulation de bas niveau
    \end{frame}

    \begin{frame}{Principe}
        On distingue deux méthodes pour l'émulation.
        \begin{enumerate}
            \item L'émulation de bas niveau (Low-level emulation)
            \begin{itemize}
                \item[\color{white}] $\hookrightarrow$ On reproduit le fonctionnement de la machine en entier.
            \end{itemize}
            \item[\color{white}] \color{white} L'émulation de haut niveau (High-level emulation)
            \begin{itemize}
                \item[\color{white}] \color{white} $\hookrightarrow$ On reproduit ce que la machine permet.
            \end{itemize}
        \end{enumerate}
        {\color{white} passage de ligne}
        
        \color{white} On s'intéresse dans un premier temps à l'émulation de bas niveau
    \end{frame}

    \begin{frame}{Principe}
        On distingue deux méthodes pour l'émulation.
        \begin{enumerate}
            \item L'émulation de bas niveau (Low-level emulation)
            \begin{itemize}
                \item[\color{white}] $\hookrightarrow$ On reproduit le fonctionnement de la machine en entier.
            \end{itemize}
            \item L'émulation de haut niveau (High-level emulation)
            \begin{itemize}
                \item[\color{white}] $\hookrightarrow$ On reproduit ce que la machine permet.
            \end{itemize}
        \end{enumerate}
        {\color{white} passage de ligne}
        
        \color{white} On s'intéresse dans un premier temps à l'émulation de bas niveau
    \end{frame}

    \begin{frame}{Principe}
        On distingue deux méthodes pour l'émulation.
        \begin{enumerate}
            \item L'émulation de bas niveau (Low-level emulation)
            \begin{itemize}
                \item[\color{white}] $\hookrightarrow$ On reproduit le fonctionnement de la machine en entier.
            \end{itemize}
            \item L'émulation de haut niveau (High-level emulation)
            \begin{itemize}
                \item[\color{white}] $\hookrightarrow$ On reproduit ce que la machine permet.
            \end{itemize}
        \end{enumerate}
        {\color{white} passage de ligne}
        
        On s'intéresse dans un premier temps à l'émulation de bas niveau
    \end{frame}

    \section{Low-level emulation}    
    \begin{frame}{Low-level emulation (LLE)}
    \end{frame}

    \appendix
    \section{Références}
    \begin{frame}{Bibliographie}
        \begin{enumerate}
            \item \guillemotleft\ Stack processor architecture and development methods suitable for dependable
applications.\ \guillemotright\ Mehdi Jallouli, Camille Diou, Fabrice Monteiro, Abbas Dandache.
            \item \guillemotleft\ Game Boy: Complete Technical Reference \guillemotright\ {\color{blue}\underline{\url{https://gekkio.fi}}}, Révision 164.
            \item L'article \guillemotleft\ Game Boy / Color Architecture - A Practical Analysis \guillemotright\ écrit par Rodrigo Copetti {\color{blue} \underline{\href{https://www.copetti.org/writings/consoles/game-boy/}{www.copetti.org/writings/consoles/game-boy/}}}.
            \item La série \guillemotleft\ The Game Boy, a hardware autopsy \guillemotright\ par JackTech {\color{blue} \underline{\url{https://www.youtube.com/@jacktech5101}}}.
        \end{enumerate}
    \end{frame}

    \begin{frame}{Ressources}
        \begin{enumerate}
            \item Architecture du processeur \guillemotleft\ Sharp SM83 \guillemotright\ {\color{blue} \underline{\url{https://gbdev.io/gb-opcodes//optables/}}}.
            \item Fichier ROM d'une cartouche de \textit{Pokémon Version Rouge} développé par Game Freak.
            \item Quelques illustrations de \guillemotleft\ Game Boy / Color Architecture - A Practical Analysis \guillemotright\ écrit par Rodrigo Copetti {\color{blue} \underline{\href{https://www.copetti.org/writings/consoles/game-boy/}{www.copetti.org/writings/consoles/game-boy/}}}.
        \end{enumerate}
    \end{frame}

    \section{Code}
    \begin{frame}{CPU.c}
    \end{frame}
    
\end{document}
